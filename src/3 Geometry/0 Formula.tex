\subsubsection{三角形内心}
	\[ \frac{a\vec {A} + b\vec{B} + c\vec{C}}{a + b + c} \]
\subsubsection{三角形外心}
	\[ \frac{\vec{A} + \vec{B} - \frac{\overrightarrow {BC} \cdot \overrightarrow{CA}}{\overrightarrow {AB} \times \overrightarrow{BC}}\overrightarrow {AB}^T}{2} \]
\subsubsection{三角形垂心}
	\[ \vec{H} = 3\vec{G} - 2\vec{O} \]
\subsubsection{三角形偏心}
	\[ \frac{-a\vec {A} + b\vec{B} + c\vec{C}}{-a + b + c} \]
	剩余两点的同理。
\subsubsection{三角形外接圆半径}
	\[ R = \frac{abc}{4S} \]
\subsubsection{Pick's theorem}
	顶点全为整点的简单多边形的面积定为$S$,边上的整点数为$B$,内部的整点数为$I$,则:
	\[ S = \frac{B}{2} + I - 1 \]
\subsubsection{超球坐标系}
\begin{eqnarray*}
 	x_1 &=& r\cos(\phi_1) \\ 
	x_2 &=& r\sin(\phi_1)\cos(\phi_2) \\
	x_3 &=& r\sin(\phi_1)\sin(\phi_2)\cos(\phi_3) \\
	\cdots\\
	x_{n-1} &=& r\sin(\phi_1)\cdots\sin(\phi_{n-2})\cos(\phi_{n-1}) \\
	x_n &=& r\sin(\phi_1)\cdots\sin(\phi_{n-2})\sin(\phi_{n-1}) \\
	\phi_{n-1} &=& 0..2*\pi\\
	\forall {i=1..{n-1}}\phi_i &=& 0..\pi\\
\end{eqnarray*}
\subsubsection{三维旋转公式}
绕着$(0,0,0)-(ux,uy,uz)$旋转$\theta$, $(ux,uy,uz)$ 是单位向量
\[
R = \begin{smallmatrix} \cos \theta +u_x^2 \left(1-\cos \theta\right) \quad u_x u_y \left(1-\cos \theta\right) - u_z \sin \theta \quad u_x u_z \left(1-\cos \theta\right) + u_y \sin \theta \\ u_y u_x \left(1-\cos \theta\right) + u_z \sin \theta \quad \cos \theta + u_y^2\left(1-\cos \theta\right) \quad u_y u_z \left(1-\cos \theta\right) - u_x \sin \theta \\ u_z u_x \left(1-\cos \theta\right) - u_y \sin \theta \quad u_z u_y \left(1-\cos \theta\right) + u_x \sin \theta \quad \cos \theta + u_z^2\left(1-\cos \theta\right) 
\end{smallmatrix}.
\]
\[
\begin{bmatrix}
x' \\
y' \\
z' \\
\end{bmatrix} = R
\begin{bmatrix}
x \\
y \\
z \\
\end{bmatrix}
\]
\subsubsection{立体角公式}
\[ \phi \] 二面角\\
\[ \Omega = \left(\phi_{ab} + \phi_{bc} + \phi_{ac}\right)\,\mathrm{rad} - \pi\,\mathrm{sr} \]\\
\[\tan \left( \frac{1}{2} \Omega/\mathrm{rad} \right) =
  \frac{\left|\vec a\ \vec b\ \vec c\right|}{abc + \left(\vec a \cdot \vec b\right)c + \left(\vec a \cdot \vec c\right)b + \left(\vec b \cdot \vec c\right)a}\]\\
\[\theta_s = \frac {\theta_a + \theta_b + \theta_c}{2}\]\\

\subsubsection{四面体体积公式}
$U, V, W, u, v, w$是四面体的$6$条棱,$U, V, W$构成三角形,$(U, u), (V, v), (W, w)$互为对棱,
则$$V = \frac{\sqrt{(s - 2a)(s - 2b)(s - 2c)(s - 2d)}}{192 uvw}$$
其中$$\left\{\begin{array}{lll}
a & = & \sqrt{xYZ}, \\
b & = & \sqrt{yZX}, \\
c & = & \sqrt{zXY}, \\
d & = & \sqrt{xyz}, \\
s & = & a + b + c + d, \\
X & = & (w - U + v)(U + v + w), \\
x & = & (U - v + w)(v - w + U), \\
Y & = & (u - V + w)(V + w + u), \\
y & = & (V - w + u)(w - u + V), \\
Z & = & (v - W + u)(W + u + v), \\
z & = & (W - u + v)(u - v + W)
\end{array}\right.$$
