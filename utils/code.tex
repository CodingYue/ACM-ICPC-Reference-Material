\section{Data Structure Forever}
\subsection{Partition Tree}
\begin{lstlisting}[language=C++]
int val[19][100100] = {0};
int lsize[19][100100] = {0};
int sorted[100100] = {0}; // [1,N], sorted needed

// build_dt(1,N)
int build_dt(int l,int r,int depth=0)
{
	if(l == r) return 0;
	int mid = (l+r)/2;
	int x = sorted[mid];
	int samecnt = mid-l+1;
	for(int i = l;i <= mid;i++) if(sorted[i] < x) samecnt--;
	
	int pl = l;
	int pr = mid+1;
	for(int i = l;i <= r;i++)
	{
		lsize[depth][i] = lsize[depth][i-1];
		if(val[depth][i] < x || (val[depth][i] == x && samecnt))
		{
			if(val[depth][i] == x) samecnt--;
			val[depth+1][pl++] = val[depth][i];
			lsize[depth][i]++;
		}
		else val[depth+1][pr++] = val[depth][i];
	}
	build_dt(l,mid,depth+1);
	build_dt(mid+1,r,depth+1);
	return 0;
}

// query_kth(1,N,l,r,k)
int query_kth(int L,int R,int l,int r,int k,int depth=0)
{
	if(l == r) return val[depth][l];
	int mid = (L+R)/2;
	int lc = lsize[depth][l-1] - lsize[depth][L-1];
	int rc = lsize[depth][r] - lsize[depth][L-1];
	int lr = l-L-lc;
	int rr = r-L-rc+1;
	if(rc - lc >= k) return query_kth(L,mid,L+lc,L+rc-1,k,depth+1);
	return query_kth(mid+1,R,mid+1+lr,mid+rr,k-(rc-lc),depth+1);
}
\end{lstlisting}
\subsection{Splay}
如果需要建初始树,记得 x->update()
\begin{lstlisting}[language=C++]
class SNode
{
public:
	int val;
	int size;
	bool rev;

	SNode* child[2];
	SNode* fa;

	int update()
	{
		pushdown();
		size = 1;
		for(int i = 0;i < 2;i++)
			if(child[i])
			{
				child[i]->pushdown();
				size += child[i]->size;
			}
		return 0;
	}
	int pushdown()
	{
		if(rev)
		{
			swap(child[0],child[1]);
			for(int i = 0;i < 2;i++)
				if(child[i]) child[i]->rev ^= 1;
			rev = false;
		}
		return 0;
	}
};

int Rotate(SNode* x,int dir)
{
	SNode* p = x->fa;
	p->pushdown();
	x->pushdown();

	p->child[dir] = x->child[dir^1];
	if(x->child[dir^1]) x->child[dir^1]->fa = p;
	x->child[dir^1] = p;

	x->fa = p->fa;
	if(!p->fa) Root = x;
	else if(p->fa->child[0] == p) p->fa->child[0] = x;
	else p->fa->child[1] = x;
	p->fa = x;
	p->update(); x->update();
	return 0;
}

SNode* Splay(SNode* x,SNode* Tar)
{
	while(x->fa != Tar)
	{
		int dir = 0;
		if(x->fa->child[0] == x) dir = 0;
		else dir = 1;
		if(x->fa->fa == Tar) Rotate(x,dir);
		else if(x->fa->fa->child[dir] == x->fa)
		{
			Rotate(x->fa,dir);
			Rotate(x,dir);
		} else {
			Rotate(x,dir);
			Rotate(x,dir^1);
		}
	}
	return x;
}

SNode* Select(SNode* x,int k)
{
	while(1)
	{
		x->pushdown();
		int xrank = 1;
		if(x->child[0]) xrank += x->child[0]->size;
		if(xrank == k) break;
		else if(k < xrank) x = x->child[0];
		else
		{
			x = x->child[1];
			k -= xrank;
		}
	}
	return x;
}
\end{lstlisting}
\subsection{BIT Kth}
\begin{lstlisting}[language=C++]
int Kth(int k)
{
	int cnt = 0;
	int ans = 0;
	for(int p = (1<<logcnt);p > 0;p >>= 1)
	{
		ans += p;
		if(ans > scorecnt || cnt+BIT[ans] >= k) ans -= p;
		else cnt += BIT[ans];
	}
	return ans+1-1;
}
\end{lstlisting}
\subsection{KD Tree}
如果被卡可以考虑写上 minx,maxx,miny,maxy 维护矩形,修改 KDTree\_Build 加上对应的维护。
\begin{lstlisting}[language=C++]
struct POINT { int x,y,id; };
inline bool cmp_x(const POINT& a,const POINT& b) { return a.x == b.x ? a.y < b.y : a.x < b.x; }
inline bool cmp_y(const POINT& a,const POINT& b) { return a.y == b.y ? a.x < b.x : a.y < b.y; }

struct KDNODE
{
	POINT p;
//	int minx,maxx,miny,maxy;

	KDNODE* Child[2];
	KDNODE* fa;
};
KDNODE NPool[111111];
KDNODE* NPTop = NPool;
KDNODE* Root;

inline KDNODE* AllocNode()
{
	memset(NPTop,0,sizeof(KDNODE));
	return NPTop++;
}

inline ll PDist(const POINT& a,const POINT& b) { return sqr((ll)(a.x-b.x))+sqr((ll)(a.y-b.y)); }

POINT pnt[111111];

KDNODE* KDTree_Build(int l,int r,int depth=0)
{
	if(l >= r) return NULL;

	if(depth&1) sort(pnt+l,pnt+r,cmp_y);
	else sort(pnt+l,pnt+r,cmp_x);

	int mid = (l+r)/2;
	KDNODE* t = AllocNode();
	
	t->Child[0] = KDTree_Build(l,mid,depth+1);
	t->Child[1] = KDTree_Build(mid+1,r,depth+1);
	for(int i = 0;i < 2;i++)
		if(t->Child[i]) t->Child[i]->fa = t;

	return t;
}

int KDTree_Insert(KDNODE* cur,POINT& P,int depth=0)
{
	KDNODE* node = AllocNode(); node->p = P;
	while(cur)
	{
		if(cur->p.x == P.x && cur->p.y == P.y && cur->p.id == P.id) break;
		int dir = 0;
		if(depth&1) dir = cmp_y(x->p,P);
		else dir = cmp_x(x->p,P);
		if(!cur->Child[dir])
		{
			cur->Child[dir] = node;
			node->fa = cur;
			break;
		}
		else
		{
			cur = cur->Child[dir];
			depth++;
		}
	}
	return 0;
}

ll KDTree_Nearest(KDNODE* x,const POINT& q,int depth=0)
{
	KDNODE* troot = x->fa;
	int dir = 0;
	while(x)
	{
		if(depth&1) dir = cmp_y(x->p,q);
		else dir = cmp_x(x->p,q);

		if(!x->Child[dir]) break;
		x = x->Child[dir];
		depth++;
	}
	ll ans = ~0ULL>>1;
	while(x != troot)
	{
		ll tans = PDist(q,x->p);
		if(tans < ans) ans = tans;
		KDNODE* oside = x->Child[dir^1];
		if(oside)
		{
			ll ldis = 0;
			/*if(depth&1) ldis = min(sqr((ll)q.y-oside->miny),sqr((ll)q.y-oside->maxy));
			else ldis = min(sqr((ll)q.x-oside->minx),sqr((ll)q.x-oside->maxx));*/
			if(depth & 1) ldis = sqr<ll>(x->p.y-q.y);
			else ldis = sqr<ll>(x->p.x-q.x);
			if(ldis < ans)
			{
				tans = KDTree_Nearest(oside,q,depth+1);
				if(tans && tans < ans) ans = tans;
			}
		}

		if(x->fa && x == x->fa->Child[0]) dir = 0;
		else dir = 1;
		x = x->fa;
		depth--;
	}
	return ans;
}
\end{lstlisting}
\subsection{Light-Heavy Decomposition}
 非递归版本, NodeID 为全局ID。
\begin{lstlisting}[language=C++]
int BlockRoot[111111];
int NodeID[111111];
int TreeSize[111111];
int Depth[111111];
int HeavyChild[111111]; // 0 if not set
int fa[111111];
int Queue[111111];
int idx = 0;
int Decomposition(int s)
{
	int qfront = 0;
	int qend = 0;
	Queue[qend++] = s;
	while(qfront < qend)
	{
		int x = Queue[qfront++];
		TreeSize[x] = 1;
		for(EDGE* e = E[x];e;e = e->Next)
		{
			int y = e->y;
			if(y == fa[x]) continue;
			
			fa[y] = x;
			Depth[y] = Depth[x]+1;
			Queue[qend++] = y;
		}
	}
	for(int i = qend-1;i >= 0;i--)
	{
		int x = Queue[i];
		for(EDGE* e = E[x];e;e = e->Next)
		{
			int y = e->y;
			if(y == fa[x]) continue;
			TreeSize[x] += TreeSize[y];
			if(TreeSize[HeavyChild[x]] < TreeSize[y]) HeavyChild[x] = y;
		}
	}

	for(int i = qend-1;i >= 0;i--)
	{
		int x = Queue[i];
		if(x == HeavyChild[fa[x]]) continue;
		int t = x;
		while(t)
		{
			BlockRoot[t] = x;
			NodeID[t] = ++idx;
			t = HeavyChild[t];
		}
	}
	return 0;
}

int ColorNode(int x,int y,int nc)
{
    while(1)
    {
        if(Depth[BlockRoot[x]] > Depth[BlockRoot[y]]) swap(x,y);
         
        if(BlockRoot[x] == BlockRoot[y])
        {
            if(Depth[x] > Depth[y]) swap(x,y);
            Seg_Modify(NodeID[x],NodeID[y],nc,1,idx);
            break;
        }
        Seg_Modify(NodeID[BlockRoot[y]],NodeID[y],nc,1,idx);
        y = fa[BlockRoot[y]];
    }
    return 0;
}
\end{lstlisting}
\subsection{Merge-Split Treap (Incomplete)}
\begin{lstlisting}[language=C++]
struct TNODE
{
	int val;
	int rd;
	int size;

	TNODE* left;
	TNODE* right;

	inline int update()
	{
		size = 1;
		if(left) size += left->size;
		if(right) size += right->size;
		return 0;
	}
};

typedef pair<TNODE*,TNODE*> ptt;

stack<TNODE*> GCPool;
TNODE TPool[11111111];
TNODE* TPTop = TPool;

TNODE* newNode(int val,int rd,TNODE* left,TNODE* right)
{
	TNODE* result = NULL;
	if(GCPool.size()) { result = GCPool.top(); GCPool.pop(); }
	else result = TPTop++;
	result->val = val; result->rd = rd; result->left = left; result->right = right;
	result->update();
	return result;
}

TNODE* Merge(TNODE* t1,TNODE* t2)
{
	if(!t1) return t2;
	if(!t2) return t1;

	if(t1->rd <= t2->rd) return newNode(t1->val,t1->rd,t1->left,Merge(t1->right,t2));
	else return newNode(t2->val,t2->rd,Merge(t1,t2->left),t2->right);
}

// split after pos nodes
ptt Split(TNODE* x,int pos)
{
	if(pos == 0) return ptt(NULL,x);
	if(pos == x->size) return ptt(x,NULL);

	int lsize = x->left ? x->left->size : 0;
	int rsize = x->right ? x->right->size : 0;
	if(lsize == pos) return ptt(x->left,x->right);
	if(pos < lsize)
	{
		ptt st = Split(x->left,pos);
		return ptt(st.first,newNode(x->val,x->rd,st.second,x->right));
	}
	else
	{
		ptt st = Split(x->right,pos-lsize-1);
		return ptt(newNode(x->val,x->rd,x->left,st.first),st.second);
	}
}
\end{lstlisting}
\subsection{XHM\_Splay}
\begin{lstlisting}[language=C++]
struct node {
	int f,ch[2],v,nl,nr,ans,s;
	node() {}
	void Init(int _v,int _f) {
		v = _v; f = _f; ch[0] = ch[1] = 0; s = abs(_v);
		nl = nr = 0; if (v > 0) nr = v; else nl = -v;
		ans = 0;
	}
}pt[MaxNode];

struct Splay {
	int root;
	void update(int t) {
		pt[t].s = pt[pt[t].ch[0]].s + pt[pt[t].ch[1]].s + abs(pt[t].v);
		pt[t].nr = max(0,pt[pt[t].ch[0]].nr + pt[t].v - pt[pt[t].ch[1]].nl) + pt[pt[t].ch[1]].nr;
		pt[t].nl = max(0,pt[pt[t].ch[1]].nl - pt[t].v - pt[pt[t].ch[0]].nr) + pt[pt[t].ch[0]].nl;
		if (pt[t].v > 0) { // node of boy
			pt[t].ans = pt[pt[t].ch[0]].ans + pt[pt[t].ch[1]].ans + min(pt[pt[t].ch[0]].nr + pt[t].v,pt[pt[t].ch[1]].nl);
		} else { // otherwise
			pt[t].ans = pt[pt[t].ch[0]].ans + pt[pt[t].ch[1]].ans + min(pt[pt[t].ch[0]].nr,pt[pt[t].ch[1]].nl - pt[t].v);
		}
	}
	void zig(int x,bool w) {
		int y = pt[x].f; if (root == y) root = x;
		pt[y].ch[!w] = pt[x].ch[w]; if (pt[x].ch[w]) pt[pt[x].ch[w]].f = y;
		pt[x].f = pt[y].f; if (root != x) pt[pt[y].f].ch[y == pt[pt[y].f].ch[1]] = x;
		pt[x].ch[w] = y; pt[y].f = x; update(y);
	}
	void splay(int x) {
		while (x != root) {
			if (pt[x].f == root) zig(x,x == pt[pt[x].f].ch[0]);
			else {
				int y = pt[x].f, z = pt[y].f;
				if (y == pt[z].ch[0]) if (x == pt[y].ch[0]) { zig(y,1); zig(x,1); } else { zig(x,0); zig(x,1); }
				else if (x == pt[y].ch[0]) { zig(x,1); zig(x,0); } else { zig(y,0); zig(x,0); }
			}
		} update(x);
	}
	void splay(int x,int f) {
		while (pt[x].f != f) {
			if (pt[pt[x].f].f == f) zig(x,x == pt[pt[x].f].ch[0]);
			else {
				int y = pt[x].f, z = pt[y].f;
				if (y == pt[z].ch[0]) if (x == pt[y].ch[0]) { zig(y,1); zig(x,1); } else { zig(x,0); zig(x,1); }
				else if (x == pt[y].ch[0]) { zig(x,1); zig(x,0); } else { zig(y,0); zig(x,0); }
			}
		} update(x);
	}
	int selFlag;
	int sel(int Key) {
		int t = root;
		while (1) {
			int ls = pt[pt[t].ch[0]].s;
			if (ls < Key && ls + abs(pt[t].v) >= Key) {
				selFlag = Key - ls;
				return t;
			}
			if (Key <= ls) t = pt[t].ch[0]; else {
				Key -= ls + abs(pt[t].v);
				t = pt[t].ch[1];
			}
		} return t;
	}
	void Del(int t) {
		while (pt[t].ch[0] + pt[t].ch[1]) if (pt[t].ch[0]) zig(pt[t].ch[0],1); else zig(pt[t].ch[1],0); 
		if (root == t) {
			root = 0; return ;
		}
		pt[pt[t].f].ch[t == pt[pt[t].f].ch[1]] = 0; splay(pt[t].f);
	}
	int bound(int x,bool w) {
		splay(x);
		int ret = pt[x].ch[w];
		while (pt[ret].ch[!w]) ret = pt[ret].ch[!w];
		return ret;
	}
	PII Split(int t,int pos) {// break node t at postion pos
		int L = bound(t,0), R = bound(t,1); Del(t);
		splay(L,0); splay(R,L); 
		int s = abs(pt[t].v); int c = (pt[t].v > 0) ? 1 : -1;
		if (pos >= 1) {
			pt[++now].Init(c * (pos),R); pt[R].ch[0] = now;
			splay(now); L = now; splay(R,L);
		}
		if (pos < abs(pt[t].v)) {
			pt[++now].Init(c * (abs(pt[t].v) - pos),R); pt[R].ch[0] = now;
			splay(now); R = now; 
		}
		return MP(L,R);
	}
}Tab;
\end{lstlisting}

\section{Graph}
\subsection{Bridge}
无向图求桥,支持重边。直接拆掉桥就是边BCC。
\begin{lstlisting}[language=C++]
int DFN[MAXN],Low[MAXN];
bool vis[MAXN],isBridge[MAXM];
int idx = 0;
int tarjan(int x,int peid=-1)
{
	vis[x] = true;
	DFN[x] = Low[x] = ++idx;
	for(EDGE* e = E[x];e;e = e->Next)
	{
		int y = e->y; int eid = e->id;
		if(eid == peid) continue;
		if(!vis[y])
		{
			tarjan(y,eid);
			Low[x] = min(Low[x],Low[y]);
		}
		else Low[x] = min(Low[x],DFN[y]);
	}
	if(peid != -1 && Low[x] == DFN[x]) isBridge[peid] = true;
	return 0;
}
\end{lstlisting}
\subsection{Cut Point}
求割点/点BCC,同样支持重边。BCCId为某条边在哪个BCC内。
\begin{lstlisting}[language=C++]
int DFN[MAXN],Low[MAXN],Stack[MAXM],BCCId[MAXM];
bool vis[MAXN],isCP[MAXN];
int idx = 0,BCCidx = 0,STop = 0;
int tarjan(int x,int peid=-1)
{
	vis[x] = true;
	DFN[x] = Low[x] = ++idx;
	int ecnt = 0;
	for(EDGE* e = E[x];e;e = e->Next)
	{
		int y = e->y; int eid = e->id;
		if(eid == peid) continue;
		if(DFN[y] < DFN[x]) Stack[STop++] = eid;
		if(!vis[y])
		{
			tarjan(y,eid);
			Low[x] = min(Low[x],Low[y]);
			ecnt++;
			if(DFN[x] <= Low[y])
			{
				BCCidx++;
				while(Stack[--STop] != e->eid) BCCId[Stack[STop]] = BCCidx;
				BCCId[e->eid] = BCCidx;

				if(peid != -1) isCP[x] = true;
			}
		}
		else Low[x] = min(Low[x],DFN[y]);
	}
	if(peid == -1 && ecnt > 1) isCP[x] = true;
	return 0;
}
\end{lstlisting}
\subsection{MMC (Karp)}
$O(nm+n^2)$ 最大平均权值环 需要存边但是不需要边表。
\begin{lstlisting}[language=C++]
int d[677][677] = {0};
double Karp(int n,int m)
{
	memset(d,0,sizeof(d));

	// init all d[0][i] with 0 if no memset or reversing 

	for(int i = 1;i <= n;i++)
		for(int j = 0;j < m;j++)
			if(d[i][E[j].y] < d[i-1][E[j].x]+E[j].k) d[i][E[j].y] = d[i-1][E[j].x]+E[j].k;
	
	double u = 0.0;
	for(int i = 0;i < n;i++)
	{
		double t = 1e100;
		for(int j = 0;j < n;j++)
		{
			if(d[j][i] >= 0)
			{
				double k = (double)(d[n][i]-d[j][i])/(n-j);
				if(k < t) t = k;
			}
		}
		if(t > u) u = t;
	}
	return u;
}
\end{lstlisting}
\subsection{LCA (Tarjan)}
$O(n)$ 仅在需要顺手维护点别的东西的时候用。
\begin{lstlisting}[language=C++]
bool vis[40000] = {0};
int djs[40000] = {0};
int djs_find(int x) { return (djs[x] == x ? x : djs[x] = djs_find(djs[x])); }

int tarjan_lca(int root)
{
	djs[root] = root;
	vis[root] = true;

	for(QLINK* i = QLink[root];i != NULL;i = i->Next)
	{
		int qx = i->q->x;
		int qy = i->q->y;
		if(qx == root && vis[qy]) i->q->lca = djs_find(qy);
		if(qy == root && vis[qx]) i->q->lca = djs_find(qx);
	}

	for(EDGE* i = E[root];i != NULL;i = i->Next)
	{
		int y = i->y;
		if(y == fa[root]) continue;
		
		tarjan_lca(y);
	}
	djs[root] = fa[root];
	return 0;
}
\end{lstlisting}
\subsection{LCA (sqr)}
 倍增LCA $O(n\log n)$ 只要维护的是树,可以动态添加 
\begin{lstlisting}[language=C++]
int fa[111111][18];
int depth[111111];
int lca(int x,int y)
{
	if(depth[x] < depth[y]) swap(x,y);
	int delta = depth[x]-depth[y];
	for(int i = 0;i < 16;i++)
	{
		if(delta&(1<<i)) x = fa[x][i];
	}
	for(int i = 15;i >= 0;i--)
	{
		if(fa[x][i] != fa[y][i]) { x = fa[x][i]; y = fa[y][i]; }
	}
	if(x != y) x = fa[x][0];
	return x;
}
int Queue[111111];
int build_lca(int root)
{
	int front = 0;
	int end = 0;
	Queue[end++] = root;
	fa[root][0] = 0; // -1 
	while(front != end)
	{
		int x = Queue[front++];
		for(EDGE* e = E[x];e;e = e->Next)
		{
			int y = e->y;
			fa[y][0] = x;
			depth[y] = depth[x]+1;
			Queue[end++] = y;	
		}
	}
	for(int i = 1;i < 18;i++)
		for(int j = 0;j < end;j++)
		{
			int x = Queue[j];
			fa[x][i] = fa[fa[x][i-1]][i-1];
		}
	return 0;

}
\end{lstlisting}
\subsection{Stable Marriage}

求的是男性最优的稳定婚姻解。稳定即没有汉子更喜欢的妹子和妹子更喜欢的汉子两情相悦的情况。男性最优即不存在所有汉子都得到了他更喜欢的妹子的解。

orderM[i][j] 为汉子i第j喜欢的妹子,
preferF[i][j] 为妹子i心中汉子j是第几位

不停的让汉子在自己的偏好列表里按顺序去找妹子,妹子取最优即可 $O(n^2)$

\begin{lstlisting}[language=C++]
int stableMarriage(int n)
{
	memset(pairM,-1,sizeof(pairM));
	memset(pairF,-1,sizeof(pairF));
	int pos[MAXN] = {0};
	for(int i = 0;i < n;i++)
	{
		while(pairM[i] == -1) // or can be implemented using queue...
		{
			int wife = orderM[i][pos[i]++];
			int ex = pairF[wife];
			if(ex == -1 || preferF[wife][i] < preferF[wife][ex])
			{
				pairM[i] = wife;
				pairF[wife] = i;

				if(ex != -1)
				{
					pairM[ex] = -1;
					i = ex; // take GREAT care
				}
			}
		}
	}
	return 0;
}
\end{lstlisting}
\subsection{Arborescence}
最小树形图,注意对 EPool 的需求是 $|V| \times |E|$ 的。
不定根的情况,造一个虚拟根, MAXINT 连上所有的点,最后答案减去 MAXINT 。
求有向森林的同上,插0边即可。可以支持负边权求最大。
\begin{lstlisting}[language=C++]
bool arborescence(int n,int root,double& ans)
{
	ans = 0;
	while(1)
	{
		double minIn[MAXN] = {0};
		int prev[MAXN] = {0};
		fill(minIn,minIn+n,MAXW);
		for(int i = 0;i < n;i++)
		{
			for(EDGE* e = E[i];e;e = e->Next)
			{
				int y = e->y;
				if(e->w < minIn[y])
				{
					minIn[y] = e->w;
					prev[y] = i;
				}
			}
		}
		for(int i = 0;i < n;i++)
		{
			for(EDGE* e = E[i];e;e = e->Next)
			{
				int y = e->y;
				if(y == root) continue;
				e->w -= minIn[e->y];
			}

			if(i == root) continue;
			if(minIn[i] == MAXW) return false; // does not exist
			ans += minIn[i];
		}
		int SCC[MAXN] = {0};
		int vis[MAXN] = {0};
		prev[root] = root;
		int sccidx = 0; int vidx = 0;
		for(int i = 0;i < n;i++)
		{
			if(vis[i]) continue;
			int x = i; vidx++;
			while(!vis[x])
			{
				vis[x] = vidx;
				SCC[x] = sccidx++;
				x = prev[x];
			}
			if(vis[x] == vidx) // circle
			{
				int ori = x;
				sccidx = SCC[x]+1;
				do 
				{
					SCC[x] = SCC[ori];
					x = prev[x];
				} while(x != ori);
			}
		}
		if(sccidx == n) break; // found
		// rebuild
		EDGE* TE[MAXN] = {0};
		for(int i = 0;i < n;i++)
		{
			for(EDGE* e = E[i];e;e = e->Next)
			{
				if(SCC[i] != SCC[e->y]) insert_edge(SCC[i],SCC[e->y],e->w,TE);
			}
		}
		memcpy(E,TE,sizeof(E));

		n = sccidx;
		root = SCC[root];
	}
	return true;
}
\end{lstlisting}
\subsection{Stoer\_Wagner}
无向图全局最小割。调用前建立邻接矩阵G,跑完后会破坏G。可记录点集。$O(n^3)$ 
\begin{lstlisting}[language=C++]
int Stoer_Wagner(int n)
{
	int mincut = 0x7FFFFFFF;
	int id[MAXN] = {0};
	int b[MAXN] = {0};
	for(int i = 0;i < n;i++) id[i] = i;
	for(;n > 1;n--)
	{
		memset(b,0,sizeof(b));
		for(int i = 0;i < n-1;i++)
		{
			int p = i+1;
			for(int j = i+1;j < n;j++)
			{
				b[id[j]] += G[id[i]][id[j]];
				if(b[id[p]] < b[id[j]]) p = j;
			}
			swap(id[i+1],id[p]);
		}
		if(b[id[n-1]] < mincut) {
			// ufs_union(st.first,st.second);
			mincut = b[id[n-1]];
			// st = pii(id[n-1],id[n-2]);
		}
		//else ufs_union(id[n-1],id[n-2]);
		for(int i = 0;i < n-2;i++)
		{
			G[id[i]][id[n-2]] += G[id[i]][id[n-1]];
			G[id[n-2]][id[i]] += G[id[n-1]][id[i]];
		}
	}
	return mincut;
}
\end{lstlisting}
\subsection{MaxFlow (ISAP)}
 最大流,时间复杂度 $O(n^2m)$ 。多次使用记得初始化。
\begin{lstlisting}[language=C++]
const int MAXM = 1000000;
const int MAXN = 25000;
const int INF = 0x7FFFFFFF;

struct ARC
{
	int y,c;
	ARC* Next,R;
};

ARC APool[MAXM*2];
ARC* APTop = APool;
ARC* Arc[MAXN];

int insert_arc(int x,int y,int c,int rc=0)
{
	ARC* fore = APTop++;
	fore->y = y; fore->c = c; fore->Next = Arc[x]; Arc[x] = fore;
	ARC* back = APTop++;
	back->y = x; back->c = rc; back->Next = Arc[y]; Arc[y] = back;

	fore->R = back; back->R = fore;
	return 0;
}

int dis[MAXN],pre[MAXN],gap[MAXN];
ARC* curArc[MAXN];
int init_distance_mark(int s,int t,int n)
{
	fill(dis,dis+MAXN,n);
	queue<int> q;
	q.push(t);
	dis[t] = 0;
	while(!q.empty())
	{
		int x = q.front(); q.pop();
		for(ARC* a = Arc[x];a;a = a->Next)
		{
			if(a->R->c <= 0) continue;
			if(dis[a->y] > dis[x]+1)
			{
				dis[a->y] = dis[x]+1;
				q.push(a->y);
			}
		}
	}
	memset(gap,0,sizeof(gap));
	for(int i = 0;i < n;i++) gap[dis[i]]++;
	return 0;
}
int max_flow(int s,int t,int n)
{
	memset(dis,0,sizeof(dis));
	memset(curArc,0,sizeof(curArc));
	// memset(gap,0,sizeof(gap));
	// gap[0] = n;
	init_distance_mark(s,t,n);

	int maxflow = 0;
	int x = s;
	while(dis[s] < n)
	{
		if(x == t)
		{
			int tflow = INF;
			while(x != s)
			{
				tflow = min(tflow,curArc[pre[x]]->c);
				x = pre[x];
			}
			x = t;
			while(x != s)
			{
				curArc[pre[x]]->c -= tflow;
				curArc[pre[x]]->R->c += tflow;
				x = pre[x];
			}
			maxflow += tflow;
			continue;
		}
		if(!curArc[x]) curArc[x] = Arc[x];
		ARC* ar = curArc[x];
		for(;ar;ar = ar->Next)
		{
			int y = ar->y;
			int c = ar->c;
			if(!c) continue;
			if(dis[y]+1 == dis[x]) break;
		}
		curArc[x] = ar;
		if(!ar)
		{
			int mindis = n+1; // relabel
			for(ARC* a = Arc[x];a;a = a->Next) if(a->c) mindis = min(mindis,dis[a->y]+1);
			gap[dis[x]]--;
			if(!gap[dis[x]]) break;
			gap[dis[x] = mindis]++;
			if(x != s) x = pre[x];
		}
		else
		{
			pre[ar->y] = x;
			x = ar->y;
		}
	}
	return maxflow;
}
\end{lstlisting}
\subsection{LT Dominator Tree}
有向图,redge是反向边。最后附有用法说明,idom是输出结果,即每个点的直接dominator点。全部标号0起始。复杂度是$O(N log N)$
\begin{lstlisting}[language=C++]
int fa[MAXN],nodeName[MAXN],nodeID[MAXN]; // ID->Name || Name->ID || ID = dfs order(DFN)
bool vis[MAXN]; int ncnt = 0;
vector<int> edges[MAXN],redges[MAXN];
int dfs(int x)
{
	vis[x] = true;
	nodeID[x] = ncnt; nodeName[ncnt++] = x;
	for(vit it = edges[x].begin();it != edges[x].end();++it)
	{
		if(vis[*it]) continue;
		fa[*it] = x; dfs(*it);
	}
	return 0;
}
int semi[MAXN],idom[MAXN],ufs[MAXN];
int mnsemi[MAXN]; // maintained during ufs_merge
vector<int> bucket[MAXN];

// x -> y
int ufs_union(int x,int y) { ufs[x] = y; return 0; }
int ufs_internal_find(int x)
{
	if(ufs[ufs[x]] == ufs[x]) return 0;
	ufs_internal_find(ufs[x]);
	if(semi[mnsemi[ufs[x]]] < semi[mnsemi[x]]) mnsemi[x] = mnsemi[ufs[x]];
	ufs[x] = ufs[ufs[x]];
	return 0;
}
int ufs_find(int x)
{
	if(ufs[x] == x) return x;
	ufs_internal_find(x);
	return mnsemi[x];
}

int calc_dominator_tree(int n)
{
	for(int i = 0;i < n;i++) { semi[i] = i; mnsemi[i] = i; ufs[i] = i; }
	for(int x = n-1;x > 0;x--)
	{
		int tfa = nodeID[fa[nodeName[x]]];
		for(vit it = redges[nodeName[x]].begin();it != redges[nodeName[x]].end();++it)
		{
			if(!vis[*it]) continue;
			int fy = ufs_find(nodeID[*it]);
			if(semi[fy] < semi[x]) semi[x] = semi[fy];
		}
		bucket[semi[x]].push_back(x);
		ufs_union(x,tfa);

		for(vit it = bucket[tfa].begin();it != bucket[tfa].end();++it)
		{
			int fy = ufs_find(*it);
			idom[nodeName[*it]] = nodeName[semi[fy] < semi[*it] ? fy : tfa];
		}
		bucket[tfa].clear();
	}
	for(int x = 1;x < n;x++)
	{
		if(idom[nodeName[x]] != nodeName[semi[x]])
		{
			idom[nodeName[x]] = idom[idom[nodeName[x]]];
		}
	}
	idom[nodeName[0]] = -1;
	return 0;
}

memset(fa,-1,sizeof(fa[0])*(n+10));
memset(idom,-1,sizeof(idom[0])*(n+10));
memset(vis,0,sizeof(vis[0])*(n+10));
for(int i = 0;i < n;i++) bucket[i].clear();
ncnt = 0;
dfs(n-1);
calc_dominator_tree(ncnt);
\end{lstlisting}
\subsection{K-short Loopless Path}
k短无环路径。邻接矩阵G存图,然后调用yenLoopless即可,s是起点,t终点,n点数,k是k。
\begin{lstlisting}[language=C++]
const int MAXN = 50;
const int INF = 0x3F3F3F3F;

class PATH
{
public:
	int node[MAXN];
	int nodecnt;
	int block[MAXN];
	int blockcnt;
	int len;
	int dev;

	PATH(int v = 0) { memset(this,0,sizeof(PATH)); node[nodecnt++] = v; }
	bool operator>(const PATH& p) const
	{
		if(len != p.len)
			return len > p.len;
		else
		{
			for(int i = p.nodecnt-1,j = nodecnt-1;i >= 0 && j >= 0;i--,j--)
			{
				if(p.node[i] != node[j]) return node[j] > p.node[i];
			}
			return nodecnt > p.nodecnt;
		}
		return false;
	}
};

int dis[MAXN];
int pre[MAXN];
int G[MAXN][MAXN];
bool vis[MAXN];

bool block[MAXN][MAXN];

// O(n^2)
int dijkstra(int n)
{
	for (int p = 0;p < n;p++)
	{
		int minV = -1;
		for (int i = 0; i < n; i++)
		{
			if (!vis[i] && (minV == -1 || dis[i] < dis[minV])) minV = i;
		}
		if (minV == -1) break;
		vis[minV] = true;

		for(int to = 0;to < n;to++)
		{
			if(!vis[to] && !block[minV][to])
			{
				int len = G[minV][to];
				if(dis[to] > dis[minV]+len || (dis[to] == dis[minV]+len && minV < pre[to]))
				{
					dis[to] = dis[minV]+len;
					pre[to] = minV;
				}
			}
		}
	}
	return 0;
}

PATH shortestPath(int v)
{
	PATH p(v);
	p.len = dis[v];
	for (v = pre[v];v != -1;v = pre[v]) p.node[p.nodecnt++] = v;
	reverse(p.node,p.node+p.nodecnt);
	return p;
}

int delSubpath(const PATH& p, int dev)
{
	int last = p.node[0];
	vis[last] = true;
	int v;
	for (int i = 1; dev != i; i++)
	{
		v = p.node[i];
		pre[v] = last;
		dis[v] = dis[last]+G[last][v];
		vis[v] = true;
		last = v;
	}
	vis[last] = false;
	return 0;
}

int initSingleSrc(int s)
{
	memset(dis,0x3F,sizeof(dis));
	memset(pre,-1,sizeof(pre));
	memset(vis,0,sizeof(vis));
	dis[s] = 0;
	return 0;
}

int yenLoopless(int s,int t,int n,int k)
{
	PATH result[201];
	int cnt = 0;

	priority_queue< PATH, vector<PATH>, greater<PATH> > candidate;
	memset(block,0,sizeof(block));
	initSingleSrc(s);
	dijkstra(n);
	if (dis[t] < INF)
	{
		PATH sh = shortestPath(t);
		sh.dev = 1;
		sh.block[sh.blockcnt++] = sh.node[sh.dev];
		candidate.push(sh);
	}
	while (cnt < k && !candidate.empty())
	{
		PATH p = candidate.top();
		candidate.pop();

		memset(block, 0, sizeof(block));
		int dev = p.dev;
		while (dev < p.nodecnt)
		{
			int last = p.node[dev-1];
			if (dev == p.dev)
			{
				for (int i = 0; i < p.blockcnt; i++)
				{
					block[last][p.block[i]] = true;
				}
			}
			else block[last][p.node[dev]] = true;

			initSingleSrc(s);
			delSubpath(p,dev);
			dijkstra(n);

			if (dis[t] < INF)
			{
				PATH newP = shortestPath(t);
				newP.dev = dev;
				if (dev == p.dev)
				{
					newP.blockcnt = p.blockcnt;
					memcpy(newP.block,p.block,sizeof(newP.block));
				}
				else newP.block[newP.blockcnt++] = p.node[dev];
				newP.block[newP.blockcnt++] = newP.node[dev];
				candidate.push(newP);
			}

			dev++;
		}
		result[cnt++] = p;
	}
	if (cnt < k) puts("No");
	else
	{
		int len = result[k-1].nodecnt;
		printf("%d",result[k-1].node[len-1]+1);
		for (int i = len-2; i >= 0; i--)
			printf("-%d",result[k-1].node[i]+1);
		putchar('\n');
	}
	return 0;
}
\end{lstlisting}

\section{Strings}
\subsection{KMP}
 求出 next 并返回 str 的循环周期。 用于匹配过程一样。
\begin{lstlisting}[language=C++]
int k_next[1111111];
int kmp(char* str,int len)
{
	int now = 0;
	for(int i = 1;i < len;i++)
	{
		while(now && str[i] != str[now]) now = k_next[now-1];
		if(str[i] == str[now]) now++;
		k_next[i] = now;
	}
	int period = len-(k_next[len-1]);
	if(len % period == 0) return period;
	return len;
}
\end{lstlisting}
\subsection{MinimalCycleExp}
 返回 text 的所有循环同构中字典序最小的起始位置。O(n) 
\begin{lstlisting}[language=C++]
int MinimalRep(char* text,int len=-1)
{
	if(len == -1) len = strlen(text);

	int i = 0;
	int j = 1;
	while(i < len && j < len)
	{
		int k = 0;
		while(k < len && text[(i+k)%len] == text[(j+k)%len]) k++;
		if(k >= len) break;

		if(text[(i+k)%len] > text[(j+k)%len]) i = max(i+k+1,j+1);
		else j = max(i+1,j+k+1);
	}
	return min(i,j);
}
\end{lstlisting}
\subsection{Gusfield}
Also known as "Extended KMP". 
  Usage:
    $z_i$ = lcp(text+i,pattern)
    Run zFunction(z\_pat,pat,pat,patLen,patLen) for self matching. 
\begin{lstlisting}[language=C++]
int z_pat[2222222] = {0};
int zFunction(int* z,char* text,char* pat,int textLen=-1,int patLen=-1)
{
	if(textLen == -1) textLen = strlen(text);
	if(patLen == -1) patLen = strlen(pat);
	
	int self = (text == pat && textLen == patLen);
	if(!self) zFunction(z_pat,pat,pat,patLen,patLen);
	else z[0] = patLen;
	
	int farfrom = 0;
	int far = self; // self->[farfrom,far) else [farfrom,far]
	for(int i = self;i < textLen;i++)
	{
		if(i+z_pat[i-farfrom] >= far)
		{
			int x = max(far,i);
			while(x < textLen && x-i < patLen && text[x] == pat[x-i]) x++;
			z[i] = x-i;
			if(i < x) { farfrom = i; far = x; }
		}
		else z[i] = z_pat[i-farfrom];
	}
	return 0;
}
\end{lstlisting}
\subsection{Aho-Corasick}
 大部分应用基于一个性质:fail指向与当前串的后缀相等的前缀最长的节点。
 另外可以模仿匹配过程在Trie上DP进行统计。
 Build a Trie then run the code below. 
\begin{lstlisting}[language=C++]
TNODE* Queue[66666];
int build_ac_automaton()
{
	int front = 0;
	int end = 0;
	Queue[end++] = Root;
	while(front != end)
	{
		TNODE* x = Queue[front++];
		for(int i = 0;i < 26;i++)
		{
			if(x->Child[i])
			{
				x->Child[i]->Fail = x->Fail?x->Fail->Child[i]:Root;
				// Spread additional info here for trie graph
				//x->Child[i]->Readable |= x->Child[i]->Fail->Readable; 
				Queue[end++] = x->Child[i];
			}
			else x->Child[i] = x->Fail?x->Fail->Child[i]:Root; // trie graph
		}
	}
	return 0;
}
\end{lstlisting}
\subsection{Manacher}
$rad_i$ 为以 i/2 为中心向两端延伸的最长回文长度。
使用 rad 时注意是按照 ab->aabb 的 Pattern 填充过的,值二倍了。
返回值为 Text 串中的最长回文长度,不需要除以 2 。
\begin{lstlisting}[language=C++]
int rad[2222222];
int Manacher(char* Text,int len)
{
	len *= 2;
	
	int k = 0;
	for(int i = 0,j = 0;i < len;i += k,j = max(j-k,0))
	{
		while((i-j)/2 >= 0 && (i+j+1)/2 < len && Text[(i-j)/2] == Text[(i+j+1)/2]) j++;
		rad[i] = j;
		for(k = 1;i-k >= 0 && rad[i]-k >= 0 && rad[i-k] != rad[i]-k;k++)
			rad[i+k] = min(rad[i-k],rad[i]-k);
	}
	return *max_element(rad,rad+len);
}
\end{lstlisting}
\subsection{Suffix Automaton}
\begin{lstlisting}[language=C++]
// Suffix Automaton //
// 自行定义 SAMNODE 结构体和相关 pool ,like a trie: child[],fa,len
SAMNODE* Root,*Last; // Must be inited!
int append_char(int ch)
{
	SAMNODE* x = Last;
	SAMNODE* t = SPTop++;
	t->len = x->len+1;
	while(x && !x->child[ch])
	{
		x->child[ch] = t;
		x = x->fa;
	}
	if(!x) t->fa = Root;
	else
	{
		SAMNODE* bro = x->child[ch];
		if(x->len+1 == bro->len) t->fa = bro; // actually it's fa.
		else
		{
			SAMNODE* nfa = SPTop++;
			nfa[0] = bro[0];
			nfa->len = x->len+1;
			bro->fa = t->fa = nfa;

			while(x && x->child[ch] == bro)
			{
				x->child[ch] = nfa;
				x = x->fa;
			}
		}
	}
	Last = t;
	return 0;
}

// SAM::Match //
SAMNODE* x = Root;
int mlen = 0;
for(int j = 0;j < len;j++)
{
	int ch = Text[j];
	/*// 强制后撤一个字符,部分情况下可能有用
	if(mlen == qlen) {
		mlen--;
		while(mlen <= x->fa->len) x = x->fa;
	} */
	if(x->child[ch]) { mlen++; x = x->child[ch]; }
	else
	{
		while(x && !x->child[ch]) x = x->fa;
		if(!x)
		{
			mlen = 0;
			x = Root;
		}
		else
		{
			mlen = x->len+1;
			x = x->child[ch];
		}
	}
	Match[j] = mlen;
} // End of SAM::Match //

// 基排方便上推一些东西,比如出现次数 //
SAMNODE* order[2222222];
int lencnt[1111111];
int post_build(int len)
{
	for(SAMNODE* cur = SPool;cur < SPTop;cur++) lencnt[cur->len]++;
	for(int i = 1;i <= len;i++) lencnt[i] += lencnt[i-1];
	int ndcnt = lencnt[len];
	for(SAMNODE* cur = SPTop-1;cur >= SPool;cur--) order[--lencnt[cur->len]] = cur;
	for(int i = ndcnt-1;i >= 0;i--) {
		// 此处上推
		if(order[i]->fa) order[i]->fa->cnt += order[i]->cnt;
	}
	return 0;
}
\end{lstlisting}
\subsection{Suffix Array}
\begin{lstlisting}[language=C++]
int aa[222222];
int ab[222222];
int* rank,last_rank,ysorted;

int sa[222222];
char Str[222222];

int cmp(int l,int r,int step)
{
	return last_rank[l] == last_rank[r] && last_rank[l+step] == last_rank[r+step];
}

int rw[222222]; 
int rsort(int n,int m)
{
	for(int i = 0;i < m;i++) rw[i] = 0;
	for(int i = 0;i < n;i++) rw[rank[ysorted[i]]]++;
	for(int i = 1;i < m;i++) rw[i] += rw[i-1];
	for(int i = n-1;i >= 0;i--) sa[--rw[rank[ysorted[i]]]] = ysorted[i]; // keep order
	return 0;
}

int da(int n,int m) // n = strlen, m = alphabet size
{
	rank = aa; last_rank = ab; ysorted = ab;
	for(int i = 0;i < n;i++) { rank[i] = Str[i]; ysorted[i] = i; }
	rsort(n,m);
	
	int p = 0; // different suffix cnt.
	for(int step = 1;p < n;step *= 2)
	{
		ysorted = last_rank; // recycle use
		
		int cnt = 0;
		for(int i = n-step;i < n;i++) ysorted[cnt++] = i;
		for(int i = 0;i < n;i++) if(sa[i] >= step) ysorted[cnt++] = sa[i]-step;
		rsort(n,m);
		
		last_rank = rank;
		rank = ysorted;
		p = 1;
		rank[sa[0]] = 0;
		for(int i = 1;i < n;i++) rank[sa[i]] = cmp(sa[i],sa[i-1],step)?p-1:p++;
		
		m = p; // take care.
	}
	return 0;
}

int height[222222]; // lcp of <TeX>suffix$_i$ and suffix$_{i-1}$</TeX>
int get_height(int n)
{
	int k = 0;
	for(int i = 0;i < n;i++)
	{
		if(rank[i] == 0) k = height[rank[i]] = 0;
		else
		{
			if(k > 0) k--;
			int j = sa[rank[i]-1];
			while(Str[i+k]==Str[j+k]) k++;
			height[rank[i]] = k;
		}
	}
	return 0;
}

int lcp(int i,int j)
{
	if(i == j) return n-i;
	if(rank[i] > rank[j]) swap(i,j);
	return rmq_querymin(rank[i]+1,rank[j]);
}
\end{lstlisting}

\section{Geometry}
\subsection{Common}
\begin{lstlisting}[language=C++]
const double eps = 1e-8;

template<typename T>
inline T valsign(T x) { return x < 0 ? -1 : (x > 0 ? 1 : 0); }
inline double valsign(double x) { return x < -eps ? -1 : (x > eps ? 1 : 0); }

// hit on the edge will return true
bool is_segment_intersect(const POINT& A,const POINT& B,const POINT& C,const POINT& D)
{
	if(max(C.x,D.x) < min(A.x,B.x) || max(C.y,D.y) < min(A.y,B.y)) return false;
	if(max(A.x,B.x) < min(C.x,D.x) || max(A.y,B.y) < min(C.y,D.y)) return false;
	if(valsign((B-A)*(C-A))*valsign((B-A)*(D-A)) > 0) return false;
	if(valsign((D-C)*(A-C))*valsign((D-C)*(B-C)) > 0) return false;
	return true;
}

POINT get_perpfoot(const POINT& LineA,const POINT& LineB,const POINT& P)
{
	if(LineA.x == LineB.x) return POINT(LineA.x,P.y);
	if(LineA.y == LineB.y) return POINT(P.x,LineA.y);
	double k = (LineA.y-LineB.y)/(LineA.x-LineB.x);
	double x = (k*(k*LineA.x+(P.y-LineA.y))+P.x)/(k*k+1.0);
	return POINT(x,k*(x-LineA.x)+LineA.y);
}

bool is_point_onseg(const POINT& LineA,const POINT& LineB,const POINT& P)
{
	if(! (min(LineA.x,LineB.x) <= P.x && P.x <= max(LineA.x,LineB.x) &&
		  min(LineA.y,LineB.y) <= P.y && P.y <= max(LineA.y,LineB.y)) )
		return false;
	if(valsign((P-LineA)*(LineB-LineA)) == 0) return true;
	return false;
}
\end{lstlisting}
\subsection{Convex Hull}
\begin{lstlisting}[language=C++]
// P is input and Hull is output.
// return point count on hull
int Graham(POINT* P,POINT* Hull,int n)
{
	sort(P,P+n);
	int HTop = 0;
	for(int i = 0;i < n;i++)
	{
		// delete collinear points
		while(HTop > 1 && valsign((P[i]-Hull[HTop-2])*(Hull[HTop-1]-Hull[HTop-2])) >= 0) HTop--;
		Hull[HTop++] = P[i];
	}
	int LTop = HTop;
	for(int i = n-2;i >= 0;i--)
	{
		while(HTop > LTop && valsign((P[i]-Hull[HTop-2])*(Hull[HTop-1]-Hull[HTop-2])) >= 0) HTop--;
		if(i) Hull[HTop++] = P[i];
	}
	return HTop;
}
\end{lstlisting}
\subsection{Euclid Nearest}
\begin{lstlisting}[language=C++]
/* Usage:
   for(int i = 0;i < N;i++) yOrder[i] = i;
   sort(P,P+N,cmp_x);
   double result = closest_pair(0,N); // Won't change array "P" */

POINT P[111111];
int yOrder[111111];
inline bool cmp_x(const POINT& a,const POINT& b)
{
	return a.x==b.x?a.y<b.y:a.x<b.x;
}

inline bool cmp_y(const int a,const int b)
{
	POINT& A = P[a];
	POINT& B = P[b];
	return A.y==B.y?A.x<B.x:A.y<B.y;
}

int thisY[111111];
// [l,r)
double closest_pair(int l,int r)
{
	double ans = 1e100;
	if(r-l <= 6)
	{
		// just brute force_-
		for(int i = l;i < r;i++)
		{
			for(int j = i+1;j < r;j++)
			{
				ans = min(ans,(P[i]-P[j]).hypot());
			}
		}
		sort(yOrder+l,yOrder+r,cmp_y);
		return ans;
	}

	int mid = (l+r)/2;
	ans = min(closest_pair(l,mid),closest_pair(mid,r));
	inplace_merge(yOrder+l,yOrder+mid,yOrder+r,cmp_y);
	
	int top = 0;
	double ll = P[mid].x;
	for(int i = l;i < r;i++)
	{
		double xx = P[yOrder[i]].x;
		if(ll-ans <= xx && xx <= ll+ans) thisY[top++] = yOrder[i];
	}

	for(int i = 0;i < top;i++)
	{
		for(int j = i+1;j < i+4 && j < top;j++)
		{
			ans = min(ans,(P[thisY[j]]-P[thisY[i]]).hypot());
		}
	}
	return ans;
}
\end{lstlisting}
\subsection{Minimal Circle Cover}
\begin{lstlisting}[language=C++]
int getcircle(POINT& a,POINT& b,POINT& c,POINT& O,double& r)
{
	double a1 = 2.0*(a.x-b.x);
	double b1 = 2.0*(a.y-b.y);
	double c1 = a.x*a.x-b.x*b.x + a.y*a.y-b.y*b.y;
	double a2 = 2.0*(a.x-c.x);
	double b2 = 2.0*(a.y-c.y);
	double c2 = a.x*a.x-c.x*c.x + a.y*a.y-c.y*c.y;
	O.x = (c1*b2-c2*b1)/(a1*b2-a2*b1);
	O.y = (c1*a2-c2*a1)/(b1*a2-b2*a1);
	r = eudis(a,O);
	return 0;
}

POINT pt[100010] = {0};

int main(void)
{
	int n = 0;
	scanf("%d",&n);
	for(int i = 0;i < n;i++) scanf("%lf %lf",&pt[i].x,&pt[i].y);
	random_shuffle(pt,pt+n);
	
	double r = 0.0;
	POINT O = pt[0];
	for(int i = 1;i < n;i++)
	{
		if(eudis(pt[i],O)-r > -eps)
		{
			O.x = (pt[0].x+pt[i].x)/2.0;
			O.y = (pt[0].y+pt[i].y)/2.0;
			r = eudis(O,pt[0]);
			for(int j = 0;j < i;j++)
			{
				if(eudis(pt[j],O)-r > -eps)
				{
					O.x = (pt[i].x+pt[j].x)/2.0;
					O.y = (pt[i].y+pt[j].y)/2.0;
					r = eudis(O,pt[i]);
					for(int k = 0;k < j;k++)
					{
						if(eudis(pt[k],O)-r > -eps)
						{
							getcircle(pt[i],pt[j],pt[k],O,r);
						}
					}
				}
			}
		}
	}
	printf("%.10f\n%.10f %.10f\n",r,O.x,O.y);
	while(getchar() != EOF);
	return 0;
}
\end{lstlisting}
\subsection{Rotate Carbin}
返回凸包上最远点对距离
\begin{lstlisting}[language=C++]
double RC(int N)
{
	double ans = 0.0;
	Hull[N] = Hull[0];
	int to = 1;
	for(int i = 0;i < N;i++)
	{
		while((Hull[i+1]-Hull[i])*(Hull[to]-Hull[i]) < (Hull[i+1]-Hull[i])*(Hull[to+1]-Hull[i])) to = (to+1)%N;
		//ans = max(ans,maxarea(i,to,N));
		ans = max(ans,(Hull[i]-Hull[to]).hypot2());
		ans = max(ans,(Hull[i+1]-Hull[to]).hypot2());
	}
	return sqrt(ans);
}
\end{lstlisting}
\subsection{Simpson}
\begin{lstlisting}[language=C++]
inline double simpson(double fl,double fr,double fmid,double l,double r) { return (fl+fr+4.0*fmid)*(r-l)/6.0; }
double rsimpson(double slr,double fl,double fr,double fmid,double l,double r)
{
	double mid = (l+r)*0.5;
	double fml = f((l+mid)*0.5);
	double fmr = f((mid+r)*0.5);
	double slm = simpson(fl,fmid,fml,l,mid);
	double smr = simpson(fmid,fr,fmr,mid,r);
	if(fabs(slr-slm-smr) < eps) return slm+smr;
	return rsimpson(slm,fl,fmid,fml,l,mid)+rsimpson(smr,fmid,fr,fmr,mid,r);
}
\end{lstlisting}

\section{Yangyue's Geometry Template}
\subsection{Common}
\begin{lstlisting}[language=C++]
int dcmp(double a, double b) {
	if (fabs(a - b) < eps) return 0;
	if (b - a > eps) return -1;
	return 1;
}
int dcmp0(double x) {
	if (x > eps) return 1;
	if (x < -eps) return -1;
	return 0;
}

struct Point {
	double x, y;
	Point(){}
	Point(double x, double y) : x(x), y(y) {}
	double operator * (const Point &b) {
		return x * b.y - y * b.x;
	}
	Point operator - (const Point &b) {
		return Point(x - b.x, y - b.y);
	}
	Point operator + (const Point &b) {
		return Point(x + b.x, y + b.y);
	}
	Point operator / (const double &t) {
		return Point(x / t, y / t);
	}
	Point operator * (const double &t) {
		return Point(x * t, y * t);
	}
	double operator %(const Point &b) {
		return x * b.x + y * b.y;
	}
	bool operator < (const Point &b) const {
		if (dcmp(x, b.x) != 0) return dcmp(x, b.x) < 0;
		return dcmp(y, b.y) < 0;
	}
	bool operator == (const Point &b) const {
		if (dcmp(x, b.x) != 0) return 0;
		if (dcmp(y, b.y) != 0) return 0;
		return 1;
	}
};
\end{lstlisting}
\subsection{Intersection}
\begin{lstlisting}[language=C++]
//判规范相交
bool stdcross(Point A, Point B, Point C, Point D) {
	int t1, t2;
	t1 = dcmp0((C - A) * (B - A));
	t2 = dcmp0((D - A) * (B - A));
	if (t1 * t2 >= 0) return 0;
	t1 = dcmp0((B - C) * (D - C));
	t2 = dcmp0((A - C) * (D - C));
	if (t1 * t2 >= 0) return 0;
	return 1;
}

//点是否在线段上
bool PinSeg(Point P, Point A, Point B) {
	if (dcmp0((P - A) * (B - A)) != 0) return 0;
	if (dcmp(P.x, min(A.x, B.x)) < 0 || dcmp(P.x, max(A.x, B.x)) > 0) return 0;
	if (dcmp(P.y, min(A.y, B.y)) < 0 || dcmp(P.y, max(A.y, B.y)) > 0) return 0;
	return 1;
}

//非规范相交
bool unstdcross(Point A, Point B, Point C, Point D) {
	if (stdcross(A, B, C, D)) return 1;
	if (PinSeg(A, C, D)) return 1;
	if (PinSeg(B, C, D)) return 1;
	if (PinSeg(C, A, B)) return 1;
	if (PinSeg(D, A, B)) return 1;
}

//相交不重合线段求交点
Point getcrossPoint(Point A, Point B, Point C, Point D) {
	if (PinSeg(A, C, D)) return A;
	if (PinSeg(B, C, D)) return B;
	if (PinSeg(C, A, B)) return C;
	if (PinSeg(D, A, B)) return D;
	double S1 = fabs((C - A) * (B - A));
	double S2 = fabs((D - A) * (B - A));
	double k = S2 / (S1 + S2);
	Point vect = C - D;
	return Point(D.x + k * vect.x, D.y + k * vect.y);
}
\end{lstlisting}
\subsection{Point3D}
\begin{lstlisting}[language=C++]
struct Point3D {
	double x, y, z;
	Point3D(){}
	Point3D(double x, double y, double z)
		: x(x), y(y), z(z) {}
	Point3D operator -(const Point3D &b) {
		return Point3D(x - b.x, y - b.y, z - b.z);
	}
	Point3D operator /(const double &t) {
		return Point3D(x / t, y / t, z / t);
	}
	Point3D operator +(const double &t) {
		return Point3D(x + t, y + t, z + t);
	}
	Point3D operator +(const Point3D &b) {
		return Point3D(x + b.x, y + b.y, z + b.z);
	}
	Point3D operator *(const double &t) {
		return Point3D(x * t, y * t, z * t);
	}
	double operator %(const Point3D &b) {
		return x * b.x + y * b.y + z * b.z;
	}
	Point3D operator *(const Point3D &b) {
		double i = y * b.z - b.y * z;
		double j = z * b.x - x * b.z;
		double k = x * b.y - b.x * y;
		return Point3D(i, j, k);
	}
	double len() {
		return sqrt( sqr(x) + sqr(y) + sqr(z) );
	}
	void init() {
		scanf("%lf%lf%lf",&x,&y,&z);
	}
};
\end{lstlisting}
\subsection{Graham}
 二维凸包 solve to b[]. 
\begin{lstlisting}[language=C++]
void graham() {
	
	sort(b, b + n);
	Point t = b[0];
	for (int i = 0; i < n; ++i) b[i] = b[i] - t;
	int m = unique(b, b + n) - b;
	//printf("%d\n", cmp(b[0], b[1]));
	
	//printf("%.10lf\n%.10lf\n", atan2(0, 0), atan2(0.70, -0.40));
	sort(b + 1, b + m, cmp);

	top = 2;
	stack[1] = 0;
	stack[2] = 1;
	for (int i = 2; i < m; ++i) {
		while (top > 1 && dcmp0(xcross2D(b[stack[top - 1]], b[stack[top]], b[i])) <= 0 ) --top;
		stack[++top] = i;
	}
}
\end{lstlisting}
\subsection{3D Convex Hull}
\begin{lstlisting}[language=C++]
struct Hull3D {
	struct Plane {
		int a, b, c;
		bool ok;
		Plane(){}
		Plane(int a, int b, int c, bool ok)
			: a(a), b(b), c(c), ok(ok) {}
	};
	int n, tricnt;        //初始点数
	int vis[MaxN][MaxN];  //点i到点j是属于哪个面
	Plane tri[MaxN << 2]; //凸包三角形
	Point3D Ply[MaxN];    //初始点
	double dist(Point3D a) {
		return sqrt(a.x * a.x + a.y * a.y + a.z * a.z); 
	}
	double area(Point3D a, Point3D b, Point3D c) {
		return dist((b - a) * (c - a));
	}
	double volume(Point3D a, Point3D b, Point3D c, Point3D d) {
		return ((b - a) * (c - a)) % (d - a);
	}
	double PtoPlane(Point3D &P, Plane f) {  // 正 : 面同向{
		Point3D m = Ply[f.b] - Ply[f.a];
		Point3D n = Ply[f.c] - Ply[f.a];
		Point3D t = P - Ply[f.a];
		return (m * n) % t;
	}
	void deal(int p, int a, int b) {
		int f = vis[a][b];
		Plane add;
		if (tri[f].ok) {
			if ((PtoPlane(Ply[p], tri[f])) > eps) dfs(p, f);
			else {
				add = Plane(b, a, p, 1);
				vis[p][b] = vis[a][p] = vis[b][a] = tricnt;
				tri[tricnt++] = add;
			}
		}
	}
	void dfs(int p, int cnt) { // 维护凸包,如果点p在凸包外更新凸包
		tri[cnt].ok = 0;
		deal(p, tri[cnt].b, tri[cnt].a);
		deal(p, tri[cnt].c, tri[cnt].b);
		deal(p, tri[cnt].a, tri[cnt].c);
	}
	bool same(int s, int e) { //判面是否相同
		Point3D a = Ply[tri[s].a];
		Point3D b = Ply[tri[s].b];
		Point3D c = Ply[tri[s].c];
		return fabs(volume(a, b, c, Ply[tri[e].a])) < eps
			&& fabs(volume(a, b, c, Ply[tri[e].b])) < eps
			&& fabs(volume(a, b, c, Ply[tri[e].c])) < eps;
	}
	void construct() { //构造凸包
		tricnt = 0;
		if (n < 4) return;
		bool tmp = 1;
		for (int i = 1; i < n; ++i) { // 两两不共点
			if (dist(Ply[0] - Ply[i]) > eps) {
				swap(Ply[1], Ply[i]);
				tmp = 0;
				break;
			}
		}
		if (tmp) return;
		tmp = 1;
		for (int i = 2; i < n; ++i) { //前三点不共线
			if ((dist((Ply[0] - Ply[1]) * (Ply[1] - Ply[i]))) > eps) {
				swap(Ply[2], Ply[i]);
				tmp = 0;
				break;
			}
		}
		if (tmp) return;
		tmp = 1;
		for (int i = 3; i < n; ++i) { //前四点不共面
			if (fabs((Ply[0] - Ply[1]) * (Ply[1] - Ply[2]) % (Ply[0] - Ply[i])) > eps) {
				swap(Ply[3], Ply[i]);
				tmp = 0;
				break;
			}
		}
		if (tmp) return;
		Plane add;
		for (int i = 0; i < 4; ++i) { //初始四面体
			add = Plane((i + 1) % 4, (i + 2) % 4, (i + 3) % 4, 1);
			if (PtoPlane(Ply[i], add) > 0) swap(add.b, add.c);
			vis[add.a][add.b] = vis[add.b][add.c] = vis[add.c][add.a] = tricnt;
			tri[tricnt++] = add;
		}
		for (int i = 4; i < n; ++i) { //构建凸包
			for (int j = 0; j < tricnt; ++j) {
				if (tri[j].ok && (PtoPlane(Ply[i], tri[j])) > eps) {
					dfs(i, j);
					break;
				}
			}
		}
		int cnt = tricnt; tricnt = 0;
		for (int i = 0; i < cnt; ++i) { //删除无用的面
			if (tri[i].ok) {
				tri[tricnt++] = tri[i];
			}
		}
	}
	int Planepolygon() { //多少个面
		int res = 0;
		for (int i = 0; i < tricnt; ++i) {
			bool yes = 1;
			for (int j = 0; j < i; ++j) {
				if (same(i, j)) {
					yes = 0;
					break;
				}
			}
			if (yes) ++res;
		}
		return res;
	}
	// Volume = sigma(volume(p, a, b, c));  i = 0..tricnt - 1;
} Hull;
\end{lstlisting}
\subsection{Halfplane}
 半平面交.. 直线的左侧
 需注意半平面的方向!
 不等式有解等价与 $cnt > 1$ 
\begin{lstlisting}[language=C++]
struct Segment {
	Point s, e;
	double angle;
	Segment(){}
	Segment(Point s, Point e)
		: s(s), e(e) {
			angle = atan2(e.y - s.y, e.x - s.x);
		}
} segment[MaxN], seg[MaxN];
Point get_intersect(Segment s1, Segment s2) {
	double u = xmul(s1.s, s1.e, s2.s);
	double v = xmul(s1.e, s1.s, s2.e);
	Point t;
	t.x = (s2.s.x * v + s2.e.x * u) / (u + v);
	t.y = (s2.s.y * v + s2.e.y * u) / (u + v);
	return t;
}
bool cmp(Segment a, Segment b) {
	if (dcmp(a.angle - b.angle) == 0) return dcmp(xmul(a.s, a.e, b.s)) > 0;
	return dcmp(a.angle - b.angle) < 0;
	return 0;
}
bool IsParallel(Segment P, Segment Q) {
	return dcmp((P.e - P.s) * (Q.e - Q.s)) == 0;
}
Segment deq[MaxN];
int HalfPlaneIntersect(Segment seg[], int n) {
	sort(seg, seg + n, cmp);
	int tmp = 1;
	for (int i = 1; i < n; ++i) {
		if (dcmp(seg[i].angle - seg[tmp - 1].angle) != 0) {
			seg[tmp++] = seg[i];
		}
	}
	n = tmp;
	//for (int i = 0; i < n; ++i) printf("%.0lf %.0lf %.0lf %.0lf\n", seg[i].s.x, seg[i].s.y, seg[i].e.x, seg[i].e.y);
	deq[0] = seg[0]; deq[1] = seg[1];
	int front = 0, tail = 1;
	for (int i = 2; i < n; ++i) {
		if(IsParallel(deq[tail], deq[tail-1]) || IsParallel(deq[front], deq[front+1])) return 0;
		while (front < tail && dcmp(xmul(seg[i].s, seg[i].e, get_intersect(deq[tail], deq[tail - 1]))) < 0) --tail;
		while (front < tail && dcmp(xmul(seg[i].s, seg[i].e, get_intersect(deq[front], deq[front+1]))) < 0) ++front;
		deq[++tail] = seg[i];
	}
	while(front < tail && xmul(deq[front].s, deq[front].e, get_intersect(deq[tail], deq[tail-1])) < -eps) tail--;
	while(front < tail && xmul(deq[tail].s, deq[tail].e, get_intersect(deq[front], deq[front+1])) < -eps) front++;
	int cnt = 0;
	deq[++tail] = deq[front];
	for (int i = front; i + 1 < tail; ++i) p[cnt++] = get_intersect(deq[i], deq[i+1]);
	return cnt;
}
\end{lstlisting}

\section{Damn Math}
\subsection{DFT}
$n$需要为2的次幂,sign传入1时正变换,传入-1时逆变换,逆变换后需要手动除以$n$。
\begin{lstlisting}[language=C++]
typedef complex<double> cplx;
inline unsigned int intrev(unsigned x)
{
	x = ((x & 0x55555555U) << 1) | ((x & 0xAAAAAAAAU) >> 1);
	x = ((x & 0x33333333U) << 2) | ((x & 0xCCCCCCCCU) >> 2);
	x = ((x & 0x0F0F0F0FU) << 4) | ((x & 0xF0F0F0F0U) >> 4);
	x = ((x & 0x00FF00FFU) << 8) | ((x & 0xFF00FF00U) >> 8);
	x = ((x & 0x0000FFFFFU) << 16) | ((x & 0xFFFF0000U) >> 16);
	return x;
};
void fft(int sign, cplx* data, int n)
{
	int d = 1+__builtin_clz(n);
	double theta = sign * 2.0 * PI / n;
	for(int m = n;m >= 2;m >>= 1, theta *= 2)
	{
		cplx tri = cplx(cos(theta),sin(theta));
		cplx w = cplx(1,0);
		for(int i = 0, mh = m >> 1; i < mh; i++)
		{
			for(int j = i;j < n;j += m)
			{
				int k = j+mh;
				cplx tmp = data[j]-data[k];
				
				data[j] += data[k];
				data[k] = w * tmp;
			}
			w *= tri;
		}
	}
	for(int i = 0;i < n;i++)
	{
		int j = intrev(i) >> d;
		if(j < i) swap(data[i],data[j]);
	}
	return;
}
\end{lstlisting}
\subsection{Linear Eratosthenes Sieve}
\begin{lstlisting}[language=C++]
int MinDivi[11111111];
int Prime[1111111];
int PCnt = 0;
int Miu[11111111];
int Phi[11111111];

int era(int N)
{
	for(int i = 2;i <= N;i++)
	{
		if(!MinDivi[i])
		{
			Prime[PCnt++] = i;
			MinDivi[i] = i;
			Miu[i] = -1;
			Phi[i] = i-1;
		}
		for(int j = 0;j < PCnt && Prime[j] <= MinDivi[i] && i*Prime[j] <= N;j++)
		{
			MinDivi[i*Prime[j]] = Prime[j];
			Miu[i*Prime[j]] = -Miu[i];
			if(Prime[j] == MinDivi[i]) Miu[i*Prime[j]] = 0;
			Phi[i*Prime[j]] = Phi[i]*(Prime[j]-(Prime[j] != MinDivi[i]));
		}
	}
	return 0;
}
\end{lstlisting}
\subsection{Miller-Rabin}
\begin{lstlisting}[language=C++]
// Always call "IsPrime" unless you know what are you doing

int MillerRabin(ull a,ull n)
{
	if(n == 2) return 1;
	if(n == 1 || (n & 1) == 0) return 0;
	ull d = n-1;
	while((d & 1) == 0) d >>= 1;
	ull t = powmod(a,d,n);
	while(d != n-1 && t != 1 && t != n-1)
	{
		t = mulmod(t,t,n);
		d <<= 1;
	}
	return (t == n-1) || ((d & 1) == 1);
}

int LPrimes[] = {2,3,5,7,11,13,17,19,23};
int IsPrime(ull n)
{
	int result = 1;
	for(int i = 0;i < sizeof(LPrimes)/sizeof(int);i++)
	{
		if(LPrimes[i] >= n) break;
		result &= MillerRabin(LPrimes[i],n);
		if(!result) return result;
	}
	return result;
}
\end{lstlisting}
\subsection{NTT}
 最终结果 mod $P$. $N \leq 200000$ 

$E_i \equiv g^{(P_i-1) \div N_1} \pmod{P_i}$ $F_i \equiv 1 \div E_i \pmod{P_i} $ $I_i \equiv 1 \div N_1 \pmod{P_i}$ 
\begin{lstlisting}[language=C++]
namespace NTT {
	const int P = 1000003; const int N = 100010; const int N1 = 262144;
	const int P1 = 998244353; const int P2 = 995622913;
	const int E1 = 996173970; const int E2 = 88560779;
	const int F1 = 121392023; const int F2 = 840835547;
	const int I1 = 998240545; const int I2 = 995619115;
	const LL M1 = 397550359381069386LL; const LL M2 = 596324591238590904LL;
	const LL MM = 993874950619660289LL;
	LL mul(LL x, LL y, LL z) {
		return (x * y - (LL) (x / (long double) z * y + 1e-3) * z + z) % z;
	}
	int crt(int x1, int x2) {
		return (mul(M1, x1, MM) + mul(M2, x2, MM)) % MM % P;
	}
	void NTT(int *A, int PM, int PW) {
		for (int m = N1, h; h = m / 2, m >= 2; PW = (LL) PW * PW % PM, m = h) {
			for (int i = 0, w = 1; i < h; ++i, w = (LL) w * PW % PM)
			for (int j = i; j < N1; j += m) {
				int k = j + h, x = (A[j] - A[k] + PM) % PM;
				A[j] += A[k]; A[j] %= PM;
				A[k] = (LL) w * x % PM;
			}
		}
		for (int i = 0, j = 1; j < N1 - 1; ++j) {
			for (int k = N1 / 2; k > (i^=k); k /= 2);
			if (j < i) swap(A[i], A[j]);
		}
	}
	int A1[MaxN], B1[MaxN], C1[MaxN];
	void mul(int *A, int *B, int *C, int n) {
		memset(C, 0, sizeof(*C)*N1);
		memcpy(A1, A, sizeof(*A)*N1);
		memcpy(B1, B, sizeof(*B)*N1);
		NTT(A1, P1, E1);
		NTT(B1, P1, E1);
		for (int i = 0; i < N1; ++i) C1[i] = (LL) A1[i] * B1[i] % P1;
		NTT(C1, P1, F1);
		for (int i = 0; i < N1; ++i) C1[i] = (LL) C1[i] * I1 % P1;
		NTT(A, P2, E2);
		NTT(B, P2, E2);
		for (int i = 0; i < N1; ++i) C[i] = (LL) A[i] * B[i] % P2;
		NTT(C, P2, F2);
		for (int i = 0; i < N1; ++i) C[i] = (LL) C[i] * I2 % P2;
		for (int i = 0; i < N1; ++i) C[i] = crt(C1[i], C[i]);
		for (int i = n; i < N1; ++i) C[i] = 0;
	}
}
\end{lstlisting}
\subsection{Pollard-Rho}
\begin{lstlisting}[language=C++]
ull PollardRho(ull n,int c)
{
	ull x = 2;
	ull y = 2;
	ull d = 1;
	//printf("%d\n",c);
	while(d == 1)
	{
		x = (mulmod(x,x,n)+c)%n;
		y = (mulmod(y,y,n)+c)%n;
		y = (mulmod(y,y,n)+c)%n;
		//printf("%I64u %I64u %I64u\n",x,y,d);

		if(x > y) d = gcd(x-y,n);
		else d = gcd(y-x,n);
	}
	return d;
}

// DO NOT CALL THIS WITH A PRIME!
ull Factorize(ull n)
{
	ull d = n;
	while(d == n) d = PollardRho(n,rand()+1);
	return d;
}

ull dv[111111];
int dvcnt = 0;

// call sort if sorted results needed.
ull FullFactorize(ull n)
{
	if(n%2 == 0)
	{
		dv[dvcnt++] = 2;
		while(n%2 == 0) n /= 2;
		return FullFactorize(n);
	}
	ull t = 0;
	while(n != 1 && !IsPrime(n))
	{
		t = Factorize(n);
		int cdvc = dvcnt;
		if(!IsPrime(t)) FullFactorize(t);
		else dv[dvcnt++] = t;
		for(int i = cdvc;i < dvcnt;i++)
		{
			while(n % dv[i] == 0) n /= dv[i];
		}
	}
	if(n != 1) dv[dvcnt++] = n;
	return 0;
}
\end{lstlisting}
\subsection{Simplex}
\begin{lstlisting}[language=C++]
// Linar programming, array all indexed from 0
double a[MaxN][MaxN], b[MaxN], c[MaxN], d[MaxN][MaxN];
int ix[MaxN + MaxN];
// max{cx|Ax<=b,x>=0}, n: constraints, m: vars
double simplex(double a[][MaxN], double b[], double c[], int n, int m) {
	++m;
	int r = n, s = m - 1;
	memset(d, 0, sizeof(d));
	for (int i = 0; i < n + m; ++i) ix[i] = i;
	for (int i = 0; i < n; ++i) {
		for (int j = 0; j < m - 1; ++j) d[i][j] = -a[i][j];
		d[i][m - 1] = 1;
		d[i][m] = b[i];
		if (d[r][m] > d[i][m]) r = i;
	}
	for (int j = 0; j < m - 1; ++j) d[n][j] = c[j];
	d[n + 1][m - 1] = -1;
	for (double dd;; ) {
		if (r < n) {
			int t = ix[s]; ix[s] = ix[r + m]; ix[r + m] = t;
			d[r][s] = 1.0 / d[r][s];
			for (int j = 0; j <= m; ++j) if (j != s) d[r][j] *= -d[r][s];
			for (int i = 0; i <= n + 1; ++i) if (i != r) {
				for (int j = 0; j <= m; ++j) if (j != s) d[i][j] += d[r][j] * d[i][s];
				d[i][s] *= d[r][s];
			}
		}
		r = -1; s = -1;
		for (int j = 0; j < m; ++j) if (s < 0 || ix[s] > ix[j]) {
			if (d[n + 1][j] > eps || (d[n + 1][j] > -eps && d[n][j] > eps)) s = j;
		}
		if (s < 0) break;
		for (int i = 0; i < n; ++i) if (d[i][s] < -eps) {
			if (r < 0 || (dd = d[r][m] / d[r][s] - d[i][m] / d[i][s]) < -eps || (dd < eps && ix[r + m] > ix[i + m])) r = i;
		}
		if (r < 0) return -1; // not bounded
	}
	if (d[n + 1][m] < -eps) return -1; // not executable
	double ans = 0;
	for (int i = m; i < n + m; ++i) {
		if (ix[i] < m - 1) ans += d[i - m][m] * c[ix[i]];
	}
	return ans;	
}
\end{lstlisting}

\section{Others}
\subsection{DLX}
\begin{lstlisting}[language=C++]
const int MAXINT = 0x7FFFFFFF;

struct DLXNODE
{
	union
	{
		int S;
		DLXNODE* C;
	};
	int Row;
	DLXNODE *U,*D,*L,*R;
};

DLXNODE H;
DLXNODE NodePool[10000] = {0};
int PoolTop = 0;

DLXNODE* node_alloc()
{
	memset(&NodePool[PoolTop],0,sizeof(DLXNODE));
	return &NodePool[PoolTop++];
}

int ans[100] = {0}; // 9x9

int remove(DLXNODE* c)
{
	c->L->R = c->R;
	c->R->L = c->L;

	for(DLXNODE* i = c->D;i != c;i = i->D)
	{
		for(DLXNODE* j = i->R;j != i;j = j->R)
		{
			j->U->D = j->D;
			j->D->U = j->U;
			j->C->S--;
		}
	}
	return 0;
}

int resume(DLXNODE* c)
{
	for(DLXNODE* i = c->D;i != c;i = i->D)
	{
		for(DLXNODE* j = i->L;j != i;j = j->L)
		{
			j->U->D = j;
			j->D->U = j;
			j->C->S++;
		}
	}

	c->L->R = c;
	c->R->L = c;
	return 0;
}

bool dfs(int k)
{
	if(H.R == &H)
	{
		// found!
		return true;
	}

	DLXNODE* tc = NULL;
	int ts = MAXINT;
	for(DLXNODE* i = H.R;i != &H;i = i->R)
	{
		if(i->S < ts)
		{
			ts = i->S;
			tc = i;
		}
	}
	if(ts == MAXINT) return true;
	remove(tc);
	for(DLXNODE* i = tc->U;i != tc;i = i->U)
	{
		ans[k] = i->Row; // store state here
		for(DLXNODE* j = i->R;j != i;j = j->R)
		{
			remove(j->C);
		}
		if(dfs(k+1)) return true;
		for(DLXNODE* j = i->L;j != i;j = j->L)
		{
			resume(j->C);
		}
	}
	resume(tc);
	return false;
}

DLXNODE* insert_to_col(DLXNODE* c,int RowNo,DLXNODE* rl)
{
	DLXNODE* node = node_alloc();
	// c->U is last node
	node->U = c->U;
	node->D = c;
	if(!rl) node->L = node->R = node;
	else
	{
		node->L = rl;
		node->R = rl->R;
		rl->R->L = node;
		rl->R = node;
	}
	node->C = c;
	node->Row = RowNo;
	c->S++;
	c->U->D = node;
	c->U = node;
	return node;
}

// 对应 9x9 数独的建图
int main(void)
{
	char Scene[100] = {0};
	while(scanf("%s",Scene) != EOF && strcmp(Scene,"end"))
	{
		PoolTop = 0;
		memset(ans,0,sizeof(ans));

		H.L = &H;
		H.R = &H;
		H.D = &H;
		H.U = &H;
		DLXNODE* cFind[324] = {0};
		DLXNODE* last = &H;
		for(int i = 0;i < 324;i++)
		{
			DLXNODE* tn = node_alloc();
			cFind[i] = tn;
			tn->S = 0;
			tn->D = tn;
			tn->U = tn;
			tn->L = last;
			tn->R = last->R;
			last->R->L = tn;
			last->R = tn;
			last = tn;
		}
		for(int i = 0;i < 9;i++)
		{
			for(int j = 0;j < 9;j++)
			{
				int s = 1; int e = 9;
				if(Scene[i*9+j] != '.')
				{
					s = e = Scene[i*9+j]-'0';
				}
				for(int k = s;k <= e;k++)
				{
					int b = (i/3)*3+j/3;

					int RowNo = i*9*9+j*9+k-1;

					DLXNODE* ln = NULL;
					ln = insert_to_col(cFind[i*9+j],RowNo,ln);
					ln = insert_to_col(cFind[81+i*9+k-1],RowNo,ln);
					ln = insert_to_col(cFind[162+j*9+k-1],RowNo,ln);
					ln = insert_to_col(cFind[243+b*9+k-1],RowNo,ln);
				}
			}
		}
		dfs(0);
		for(int i = 0;i < 81;i++)
		{
			int RNo = ans[i];
			int k = RNo % 9 + 1;
			int j = RNo / 9 % 9;
			int r = RNo / 81;
			Scene[r*9+j] = '0' + k;
		}
		printf("%s\n",Scene);
	}
	return 0;
}
\end{lstlisting}
\subsection{FastIO For Java}
\begin{lstlisting}[language=Java]
BufferedReader reader = new BufferedReader(new InputStreamReader(System.in));
StringTokenizer tokenizer = null;

public String next()
{
	while (tokenizer == null || !tokenizer.hasMoreTokens()) {
		try {
			tokenizer = new StringTokenizer(reader.readLine());
		} catch(IOException e) {
			throw new RuntimeException(e);
		}
	}
	return tokenizer.nextToken();
}

public int nextInt() {
	return Integer.parseInt(next()); // Double. .....
}
\end{lstlisting}
\subsection{Java References}
有一个叫DecimalFormat的东西。

有一个叫BufferedInputStream的东西。

有一个叫FileInputStream的东西。

\section{外挂}
\subsection{mulmod}
\begin{lstlisting}[language=C++]
/* return x*y%mod. no overflow if x,y < mod
 * remove 'i' in "idiv"/"imul" if change to unsigned*/
inline int mulmod(int x,int y,int mod)
{
	int ans = 0;
	__asm__
	(
		"movl %1,%%eax\n"
		"imull %2\n"
		"idivl %3\n"
		
		:"=d"(ans)
		:"m"(x),"m"(y),"m"(mod)
		:"%eax"
	);
	return ans;
}
\end{lstlisting}
\subsection{stack}
修改 esp 到手动分配的内存。
慎用!可能违反某些规则或造成不必要的 RE/WA 。
\begin{lstlisting}[language=C++]
int main(void)
{
	char* SysStack = NULL;
	char* MyStack = new char[33554432];
	MyStack += 33554432-1048576; // 32M
	__asm__(
		"movl %%esp,%%eax\n\t"
		"movl %1,%%esp\n\t"
		:"=a"(SysStack)
		:"m"(MyStack)
	);
	mmain();
	__asm__(
		"movl %0,%%esp\n\t"
		::"m"(SysStack)
	);
	return 0;
}
\end{lstlisting}
